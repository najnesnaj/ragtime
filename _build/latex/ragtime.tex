%% Generated by Sphinx.
\def\sphinxdocclass{report}
\documentclass[letterpaper,10pt,english]{sphinxmanual}
\ifdefined\pdfpxdimen
   \let\sphinxpxdimen\pdfpxdimen\else\newdimen\sphinxpxdimen
\fi \sphinxpxdimen=.75bp\relax
\ifdefined\pdfimageresolution
    \pdfimageresolution= \numexpr \dimexpr1in\relax/\sphinxpxdimen\relax
\fi
%% let collapsible pdf bookmarks panel have high depth per default
\PassOptionsToPackage{bookmarksdepth=5}{hyperref}

\PassOptionsToPackage{booktabs}{sphinx}
\PassOptionsToPackage{colorrows}{sphinx}

\PassOptionsToPackage{warn}{textcomp}
\usepackage[utf8]{inputenc}
\ifdefined\DeclareUnicodeCharacter
% support both utf8 and utf8x syntaxes
  \ifdefined\DeclareUnicodeCharacterAsOptional
    \def\sphinxDUC#1{\DeclareUnicodeCharacter{"#1}}
  \else
    \let\sphinxDUC\DeclareUnicodeCharacter
  \fi
  \sphinxDUC{00A0}{\nobreakspace}
  \sphinxDUC{2500}{\sphinxunichar{2500}}
  \sphinxDUC{2502}{\sphinxunichar{2502}}
  \sphinxDUC{2514}{\sphinxunichar{2514}}
  \sphinxDUC{251C}{\sphinxunichar{251C}}
  \sphinxDUC{2572}{\textbackslash}
\fi
\usepackage{cmap}
\usepackage[T1]{fontenc}
\usepackage{amsmath,amssymb,amstext}
\usepackage{babel}



\usepackage{tgtermes}
\usepackage{tgheros}
\renewcommand{\ttdefault}{txtt}



\usepackage[Bjarne]{fncychap}
\usepackage{sphinx}

\fvset{fontsize=auto}
\usepackage{geometry}


% Include hyperref last.
\usepackage{hyperref}
% Fix anchor placement for figures with captions.
\usepackage{hypcap}% it must be loaded after hyperref.
% Set up styles of URL: it should be placed after hyperref.
\urlstyle{same}

\addto\captionsenglish{\renewcommand{\contentsname}{Contents:}}

\usepackage{sphinxmessages}
\setcounter{tocdepth}{1}



\title{ragtime}
\date{Dec 27, 2024}
\release{1}
\author{jan jansen}
\newcommand{\sphinxlogo}{\vbox{}}
\renewcommand{\releasename}{Release}
\makeindex
\begin{document}

\ifdefined\shorthandoff
  \ifnum\catcode`\=\string=\active\shorthandoff{=}\fi
  \ifnum\catcode`\"=\active\shorthandoff{"}\fi
\fi

\pagestyle{empty}
\sphinxmaketitle
\pagestyle{plain}
\sphinxtableofcontents
\pagestyle{normal}
\phantomsection\label{\detokenize{index::doc}}


\sphinxstepscope


\chapter{Intro}
\label{\detokenize{intro:intro}}\label{\detokenize{intro::doc}}
\sphinxAtStartPar
These are some building blocks in what should become a local LLM for financial info.
I do like investing, but I rather not read all the publications.

\sphinxAtStartPar
This software should be able to  :
\begin{itemize}
\item {} 
\sphinxAtStartPar
leech reports

\item {} 
\sphinxAtStartPar
embed them into a vectordb

\item {} 
\sphinxAtStartPar
use a local LLM to retrieve and bundle information

\item {} 
\sphinxAtStartPar
generate a report

\end{itemize}


\section{Homelab: (did not want to create separate repo)}
\label{\detokenize{intro:homelab-did-not-want-to-create-separate-repo}}
\sphinxAtStartPar
I have an old HP dl380p gen8, which I modified:
\sphinxhyphen{} removed a SAS controller card, and have disks on internal controller
\sphinxhyphen{} add a 1TB NVME M2 disk, on PCI\sphinxhyphen{}e 3 adapter
\sphinxhyphen{} installed Proxmox

\sphinxAtStartPar
todo:
\sphinxhyphen{} boot on sata disk on cdrom port
\sphinxhyphen{} remove RAID1 and use disks as such: space over security
\sphinxhyphen{} insert NVIDIA P4 single slot

\sphinxstepscope


\chapter{readthedocs}
\label{\detokenize{readthedocs:readthedocs}}\label{\detokenize{readthedocs::doc}}\begin{itemize}
\item {} 
\sphinxAtStartPar
use restructured text

\end{itemize}


\section{sphinx}
\label{\detokenize{readthedocs:sphinx}}
\sphinxAtStartPar
pip install sphinx
sphinx\sphinxhyphen{}quickstart


\section{integration github}
\label{\detokenize{readthedocs:integration-github}}
\sphinxAtStartPar
\sphinxurl{https://readthedocs.org/dashboard/}

\sphinxAtStartPar
\sphinxstyleemphasis{include .readthedocs.yml in github repository}

\begin{sphinxVerbatim}[commandchars=\\\{\}]
version:\PYG{+w}{ }\PYG{l+m}{2}

build:
\PYG{+w}{   }os:\PYG{+w}{ }ubuntu\PYGZhy{}22.04
\PYG{+w}{   }tools:
\PYG{+w}{      }python:\PYG{+w}{ }\PYG{l+s+s2}{\PYGZdq{}3.12\PYGZdq{}}

sphinx:
\PYG{+w}{   }configuration:\PYG{+w}{ }conf.py
\end{sphinxVerbatim}

\sphinxstepscope


\chapter{install something from docker}
\label{\detokenize{docker:install-something-from-docker}}\label{\detokenize{docker::doc}}
\sphinxAtStartPar
\sphinxstyleemphasis{the docker build that comes with the apt\sphinxhyphen{}install from ubuntu does not always cut the cake}
\begin{quote}

\sphinxAtStartPar
\sphinxurl{https://docs.docker.com/engine/install/ubuntu/}
\sphinxurl{https://download.docker.com/linux/ubuntu/dists/jammy/pool/stable/amd64/}
\end{quote}


\section{using portainer:}
\label{\detokenize{docker:using-portainer}}\begin{quote}

\sphinxAtStartPar
sudo docker volume create portainer\_data
sudo docker run \sphinxhyphen{}d \sphinxhyphen{}p 8000:8000 \sphinxhyphen{}p 9443:9443 \textendash{}name portainer \textendash{}restart=always \sphinxhyphen{}v /var/run/docker.sock:/var/run/docker.sock \sphinxhyphen{}v portainer\_data:/data portainer/portainer\sphinxhyphen{}ce:2.21.3
\end{quote}


\section{using github to build containers:}
\label{\detokenize{docker:using-github-to-build-containers}}
\sphinxAtStartPar
see github actions (and actions.rst in the doc)


\section{Project Dockerfiles:}
\label{\detokenize{docker:project-dockerfiles}}
\sphinxAtStartPar
the project dockerfiles are within their own directory:
\sphinxhyphen{} leecher
\sphinxhyphen{} embedder
\sphinxhyphen{} postgres


\section{leecher:}
\label{\detokenize{docker:leecher}}
\sphinxAtStartPar
this container waits for an incoming file in the dockervolume : and then converts pdf to txt


\section{embedder:}
\label{\detokenize{docker:embedder}}
\sphinxAtStartPar
this container chops the txt\sphinxhyphen{}file and inserts into a postgres database


\section{postgres:}
\label{\detokenize{docker:postgres}}
\sphinxAtStartPar
this a combination of management and a vectordatabase
(contains the script to create the ‘document\_chunks’ table


\section{Containers}
\label{\detokenize{docker:containers}}\begin{itemize}
\item {} 
\sphinxAtStartPar
the Dockerfiles help to build docker images

\item {} 
\sphinxAtStartPar
the docker compose file help to build containers

\end{itemize}


\section{compose}
\label{\detokenize{docker:compose}}
\sphinxAtStartPar
there are docker\sphinxhyphen{}compose.yml files in :
\sphinxhyphen{} postgres
\sphinxhyphen{} rag

\sphinxAtStartPar
The one in postgres :
\sphinxhyphen{} creates 2 containers  (database + management)
\sphinxhyphen{} inits the postgres database as a vectordatabase and creates a table

\sphinxAtStartPar
The one in rag :
\sphinxhyphen{} creates a volume where you can copy pdf files
\sphinxhyphen{} creates a volume where converted text files are stored
\sphinxhyphen{} defines environment variables to access the database (to be changed on your environment!)
\sphinxstyleemphasis{docker compose up \sphinxhyphen{}d}


\section{how to copy pdf files to the container?}
\label{\detokenize{docker:how-to-copy-pdf-files-to-the-container}}\begin{itemize}
\item {} 
\sphinxAtStartPar
the dockervolumes are created using the docker\sphinxhyphen{}compose files

\item {} 
\sphinxAtStartPar
dockervolumes link a directory in docker to a directory on the filesystem

\end{itemize}

\sphinxAtStartPar
cp 270123.pdf /var/snap/docker/common/var\sphinxhyphen{}lib\sphinxhyphen{}docker/volumes/rag\_leech\_data/\_data


\section{using github actions to build docker containers}
\label{\detokenize{docker:using-github-actions-to-build-docker-containers}}
\sphinxAtStartPar
each time there is is a push toward the github repository, automatically a build of the docker images gets triggered.

\sphinxAtStartPar
I use multiple Dockerfiles, thus multiple Docker images, and I couldn’t not figure out the easy\sphinxhyphen{}way how to build them with a single script.

\sphinxAtStartPar
So … multiple scripts, which each build a single image.

\sphinxAtStartPar
By default the image gets a name like this repo:main.
This can be modified!


\chapter{TRICK : multiple containers with github}
\label{\detokenize{docker:trick-multiple-containers-with-github}}\begin{itemize}
\item {} 
\sphinxAtStartPar
copy docker\sphinxhyphen{}publish.yml to docker\sphinxhyphen{}publish2.yml

\item {} 
\sphinxAtStartPar
change IMAGE\_NAME: ‘najnesnaj/embed’

\end{itemize}

\sphinxAtStartPar
(./embedder is de directory in the repo that contains the Dockerfile)
\begin{description}
\sphinxlineitem{and change :}\begin{description}
\sphinxlineitem{\# Build and push Docker image for embedder}\begin{itemize}
\item {} 
\sphinxAtStartPar
name: Build and push Docker image (embedder)
id: build\sphinxhyphen{}and\sphinxhyphen{}push\sphinxhyphen{}embedder
uses: \sphinxhref{mailto:docker/build-push-action@0565240e2d4ab88bba5387d719585280857ece09}{docker/build\sphinxhyphen{}push\sphinxhyphen{}action@0565240e2d4ab88bba5387d719585280857ece09} \# v5.0.0
with:
\begin{quote}

\sphinxAtStartPar
context: ./embedder
\end{quote}

\end{itemize}

\end{description}

\end{description}

\sphinxAtStartPar
first login to github

\sphinxAtStartPar
echo “ghp\_xxxxxxxxxxxxxxxxxxxxxxxxxxxxxxxxxzK” | docker login ghcr.io \sphinxhyphen{}u najnesnaj \textendash{}password\sphinxhyphen{}stdin

\sphinxAtStartPar
Login Succeeded

\sphinxAtStartPar
now I can download the image
\sphinxhref{mailto:naj@naj-Latitude-5520}{naj@naj\sphinxhyphen{}Latitude\sphinxhyphen{}5520}:/usr/src/ragtime\$ sudo docker pull ghcr.io/najnesnaj/embed:main


\section{elasticsearch:}
\label{\detokenize{docker:elasticsearch}}\begin{itemize}
\item {} 
\sphinxAtStartPar
sudo docker run \sphinxhyphen{}d \textendash{}name elasticsearch \sphinxhyphen{}p 9200:9200 \sphinxhyphen{}e “discovery.type=single\sphinxhyphen{}node” docker.elastic.co/elasticsearch/elasticsearch:8.15.3

\item {} 
\sphinxAtStartPar
sudo docker run \textendash{}name kibana \sphinxhyphen{}p 5601:5601 \textendash{}link elasticsearch:elasticsearch kibana:8.15.3

\end{itemize}

\sphinxAtStartPar
Kibana has not been configured.

\sphinxAtStartPar
Go to \sphinxurl{http://0.0.0.0:5601/?code=003763} to get started.

\sphinxAtStartPar
in the elasticsearch container generate token
(bin/elasticsearch\sphinxhyphen{}create\sphinxhyphen{}enrollment\sphinxhyphen{}token \sphinxhyphen{}s kibana)
elasticsearch\sphinxhyphen{}users useradd test
elasticsearch\sphinxhyphen{}users passwd test
elasticsearch\sphinxhyphen{}users roles \sphinxhyphen{}a kibana\_admin test

\sphinxstepscope


\chapter{Elasticsearch}
\label{\detokenize{elastic:elasticsearch}}\label{\detokenize{elastic::doc}}
\sphinxAtStartPar
for installation see: docker

\sphinxAtStartPar
\sphinxurl{https://www.elastic.co/guide/en/elasticsearch/reference/current/semantic-search-elser.html}

\sphinxAtStartPar
\sphinxurl{https://www.elastic.co/search-labs/tutorials/examples}

\sphinxstepscope


\chapter{storing vector + metadata}
\label{\detokenize{pgvector:storing-vector-metadata}}\label{\detokenize{pgvector::doc}}
\sphinxAtStartPar
PostgreSQL + pgvector Example
If you prefer using PostgreSQL, you can use the pgvector extension, which allows you to store vector embeddings in a PostgreSQL table alongside metadata.

\sphinxAtStartPar
Steps:
\begin{itemize}
\item {} 
\sphinxAtStartPar
Install pgvector: First, install the pgvector extension.

\item {} 
\sphinxAtStartPar
Create a table: Create a table that stores both vectors and metadata.

\item {} 
\sphinxAtStartPar
Insert vectors and metadata: Insert each chunk’s vector along with its metadata.

\end{itemize}

\sphinxAtStartPar
sql

\sphinxstepscope


\chapter{Github Actions}
\label{\detokenize{actions:github-actions}}\label{\detokenize{actions::doc}}

\section{build docker images in github}
\label{\detokenize{actions:build-docker-images-in-github}}
\sphinxAtStartPar
//github.com/najnesnaj/ragtime/actions/new

\sphinxAtStartPar
Continuous integration
\begin{itemize}
\item {} 
\sphinxAtStartPar
Publish Docker Container

\item {} 
\sphinxAtStartPar
Build, test and push Docker image to GitHub Packages.

\end{itemize}


\section{docker publish}
\label{\detokenize{actions:docker-publish}}
\sphinxAtStartPar
in \sphinxurl{https://github.com/najnesnaj/ragtime/tree/main/.github/workflows},
docker\sphinxhyphen{}publish.yml is added


\section{adapt docker\sphinxhyphen{}publish.yml}
\label{\detokenize{actions:adapt-docker-publish-yml}}
\sphinxAtStartPar
Dockerfile definitions in this repository are to be found in subdirectories
the docker\sphinxhyphen{}publish script has to be made aware


\section{where to find the published containers?}
\label{\detokenize{actions:where-to-find-the-published-containers}}
\sphinxAtStartPar
\sphinxurl{https://github.com/najnesnaj/ragtime/pkgs/container/ragtime}

\sphinxstepscope

\sphinxAtStartPar
version: ‘3’

\sphinxAtStartPar
services:
\begin{quote}

\sphinxAtStartPar
nginx:
\begin{quote}

\sphinxAtStartPar
image: nginx:latest

\sphinxAtStartPar
container\_name: nginx\_proxy

\sphinxAtStartPar
ports:
\begin{itemize}
\item {} 
\sphinxAtStartPar
“9443:9443”

\end{itemize}

\sphinxAtStartPar
volumes:
\begin{itemize}
\item {} 
\sphinxAtStartPar
./nginx.conf:/etc/nginx/nginx.conf

\item {} 
\sphinxAtStartPar
/etc/letsencrypt:/etc/letsencrypt  \# Mount certs

\end{itemize}

\sphinxAtStartPar
networks:
\begin{itemize}
\item {} 
\sphinxAtStartPar
proxy

\end{itemize}
\end{quote}

\sphinxAtStartPar
certbot:
\begin{quote}

\sphinxAtStartPar
image: certbot/certbot

\sphinxAtStartPar
container\_name: certbot

\sphinxAtStartPar
command: certonly \textendash{}webroot \textendash{}webroot\sphinxhyphen{}path=/var/www/certbot \sphinxhyphen{}d www.melborp.solutions

\sphinxAtStartPar
volumes:
\begin{itemize}
\item {} 
\sphinxAtStartPar
/etc/letsencrypt:/etc/letsencrypt  \# Persist certs

\item {} 
\sphinxAtStartPar
/var/www/certbot:/var/www/certbot

\end{itemize}

\sphinxAtStartPar
networks:
\begin{itemize}
\item {} 
\sphinxAtStartPar
proxy

\end{itemize}
\end{quote}

\sphinxAtStartPar
app:
\begin{quote}

\sphinxAtStartPar
image: your\_app\_image

\sphinxAtStartPar
container\_name: your\_app

\sphinxAtStartPar
expose:
\begin{itemize}
\item {} 
\sphinxAtStartPar
“8080”  \# Internal port

\end{itemize}

\sphinxAtStartPar
networks:
\begin{itemize}
\item {} 
\sphinxAtStartPar
proxy

\end{itemize}
\end{quote}
\end{quote}

\sphinxAtStartPar
networks:
\begin{quote}

\sphinxAtStartPar
proxy:
\begin{quote}

\sphinxAtStartPar
external: true
\end{quote}
\end{quote}

\sphinxstepscope


\chapter{python technology}
\label{\detokenize{technology:python-technology}}\label{\detokenize{technology::doc}}
\sphinxAtStartPar
A simple application was created with the help of chatgpt.


\chapter{application:}
\label{\detokenize{technology:application}}\begin{itemize}
\item {} 
\sphinxAtStartPar
upload pdf

\item {} 
\sphinxAtStartPar
convert pdf

\item {} 
\sphinxAtStartPar
correct text

\item {} 
\sphinxAtStartPar
split text

\end{itemize}


\chapter{technology}
\label{\detokenize{technology:technology}}\begin{itemize}
\item {} 
\sphinxAtStartPar
Flask  : created html templates, which got javascript to handle text, for me: too complex

\item {} 
\sphinxAtStartPar
FastAPI : similar to Flask

\item {} 
\sphinxAtStartPar
streamlit : not bothered by html, small python application

\item {} 
\end{itemize}

\sphinxstepscope


\chapter{Welcome to proxmox\sphinxhyphen{}dox’s documentation!}
\label{\detokenize{proxmox/index:welcome-to-proxmox-dox-s-documentation}}\label{\detokenize{proxmox/index::doc}}
\sphinxstepscope


\section{intro}
\label{\detokenize{proxmox/intro:intro}}\label{\detokenize{proxmox/intro::doc}}
\sphinxAtStartPar
This doc does not belong here, but did not want to create multiple github repo’s.

\sphinxAtStartPar
This is a personal account of venturing into the Proxmox virutualised world:
\begin{itemize}
\item {} 
\sphinxAtStartPar
running containers

\item {} 
\sphinxAtStartPar
running local LLM

\item {} 
\sphinxAtStartPar
no GPU

\item {} 
\sphinxAtStartPar
cheap

\end{itemize}

\sphinxAtStartPar
I try to document as much as I can, to avoid solving the same problem twice.


\subsection{homelab}
\label{\detokenize{proxmox/intro:homelab}}
\sphinxAtStartPar
I bought hp dl380p g8 (generation8) with 16 cores and 32 threads and 256 GIGA bytes of RAM!! for the price of a raspberry pi.

\sphinxAtStartPar
I plan it on using during winter, so its heat is not lost. During rendering or running a LLM it generates 400 Watt. (or consumes for 400 watt expensive electricity)

\sphinxAtStartPar
It could use a GPU, but I hate to spend more money, and it would need a special riser to give way to PCIe x16 double slot.

\sphinxAtStartPar
results sofar :
\begin{itemize}
\item {} 
\sphinxAtStartPar
LLM runs at 5 tokens / second (llavafile)

\item {} 
\sphinxAtStartPar
blender takes half an hour for rendering (CYCLES) a single picture

\item {} 
\sphinxAtStartPar
using it as a  ramdisk give 1,5G/s readspead!

\end{itemize}


\subsection{software}
\label{\detokenize{proxmox/intro:software}}
\sphinxAtStartPar
I choose proxmox as a virtualisation platform, it runs linux containers (LXC), which are kind of cool since they are created quickly, launched quickly. Some problems arise since there are still shared resources with the host… hence the use of the included templates

\sphinxstepscope


\section{Upgrading homelab HP DL 380p}
\label{\detokenize{proxmox/hpdl380p:upgrading-homelab-hp-dl-380p}}\label{\detokenize{proxmox/hpdl380p::doc}}

\subsection{latest BIOS}
\label{\detokenize{proxmox/hpdl380p:latest-bios}}\begin{itemize}
\item {} 
\sphinxAtStartPar
download latest firmware (BIOS) for linux in RPM format

\item {} 
\sphinxAtStartPar
firmware\sphinxhyphen{}system\sphinxhyphen{}p70\sphinxhyphen{}2019.05.24\sphinxhyphen{}1.1

\item {} 
\sphinxAtStartPar
on ubuntu (unpack with Ark)

\item {} 
\sphinxAtStartPar
look for CPQP7013.6B8 (4MB in size)

\item {} 
\sphinxAtStartPar
use ilo to upload firmware (update)

\end{itemize}


\subsection{M2 disk drive}
\label{\detokenize{proxmox/hpdl380p:m2-disk-drive}}\begin{itemize}
\item {} 
\sphinxAtStartPar
M.2 NGFF SSD naar PCI\sphinxhyphen{}E 3.0 X16 High\sphinxhyphen{}Speed SSD  (ashata = 5euro)

\item {} 
\sphinxAtStartPar
need MVME (one notch) m2 card

\item {} 
\sphinxAtStartPar
it shows up in BIOS / PCI devices

\item {} 
\sphinxAtStartPar
on linux :\#lsblk it should show up

\end{itemize}


\subsection{Sata}
\label{\detokenize{proxmox/hpdl380p:sata}}\begin{itemize}
\item {} 
\sphinxAtStartPar
m2 to sata adapter case (aliexpress)

\item {} 
\sphinxAtStartPar
cable female sata to female Slimline 13pin 7 + 6 (aliexpress)

\item {} 
\sphinxAtStartPar
m2 sata disk 128G

\end{itemize}


\subsection{Bootconfig}
\label{\detokenize{proxmox/hpdl380p:bootconfig}}
\sphinxAtStartPar
there is a
\sphinxhyphen{} SAS controller
\sphinxhyphen{} SATA controller (cdrom)

\sphinxAtStartPar
the controllor has a bootorder as well !!!!
In order to boot from sata, the sata controller has to boot first!!!

\sphinxAtStartPar
{\color{red}\bfseries{}**}this is the way to boot an expensive hp server from a simple 2,5 inch sata disk (laptop 5V) ,using the onboard slimline cd\sphinxhyphen{}rom connector

\sphinxstepscope


\section{Where can I find more Templates?}
\label{\detokenize{proxmox/templates:where-can-i-find-more-templates}}\label{\detokenize{proxmox/templates::doc}}\begin{itemize}
\item {} 
\sphinxAtStartPar
Use:

\end{itemize}

\sphinxAtStartPar
pveam update
\begin{itemize}
\item {} 
\sphinxAtStartPar
to update the container template database, then:

\end{itemize}

\sphinxAtStartPar
pveam available

\sphinxstepscope


\section{reading data from USB stick}
\label{\detokenize{proxmox/usb:reading-data-from-usb-stick}}\label{\detokenize{proxmox/usb::doc}}
\sphinxAtStartPar
the USB stick is readable from the proxmox host:

\sphinxAtStartPar
(do a dmesg to get the device: in this case /dev/sdb1)
mount /dev/sdb1 /usbdrive


\subsection{create a mp on CT (container02)}
\label{\detokenize{proxmox/usb:create-a-mp-on-ct-container02}}
\sphinxAtStartPar
/dev/mapper/pve\sphinxhyphen{}vm\textendash{}102\textendash{}disk\textendash{}1  51290592        28  48652740   1\% /container02mp


\subsection{transfer to CT (name = container02)}
\label{\detokenize{proxmox/usb:transfer-to-ct-name-container02}}
\sphinxAtStartPar
on the host
mkdir /drive\_container02
mount /dev/mapper/pve\sphinxhyphen{}vm\textendash{}102\textendash{}disk\textendash{}1 /drive\_container02/

\sphinxstepscope


\section{ramdisk}
\label{\detokenize{proxmox/ramdisk:ramdisk}}\label{\detokenize{proxmox/ramdisk::doc}}
\sphinxAtStartPar
My dl380p gen8 has 256Gb of RAM


\subsection{Can I use RAM as a disk?}
\label{\detokenize{proxmox/ramdisk:can-i-use-ram-as-a-disk}}
\sphinxAtStartPar
mkdir /tmp/ramdisk
chmod 777 /tmp/ramdisk
mount \sphinxhyphen{}t tmpfs \sphinxhyphen{}o size=1024m myramdisk /tmp/ramdisk

\sphinxAtStartPar
or to get something extra…

\sphinxAtStartPar
sudo mount \sphinxhyphen{}t tmpfs \sphinxhyphen{}o size=10G myramdisk /tmp/ramdisk


\subsection{speedtest}
\label{\detokenize{proxmox/ramdisk:speedtest}}
\sphinxAtStartPar
For Write:
dd if=/dev/zero of=/dev/shm/ram bs=1048576 count=4096 oflag=nocache conv=fsync
4096+0 records in
4096+0 records out
4294967296 bytes (4.3 GB, 4.0 GiB) copied, 2.79948 s, 1.5 GB/s

\sphinxAtStartPar
or:
dd if=/dev/zero of=/tmp/ramdisk/blok bs=1048576 count=1024 oflag=nocache conv=fsync
1024+0 records in
1024+0 records out
1073741824 bytes (1.1 GB, 1.0 GiB) copied, 0.560324 s, 1.9 GB/s

\sphinxAtStartPar
For Read:

\sphinxAtStartPar
dd if=/tmp/ramdisk/blok of=/dev/null bs=1048576 iflag=nocache,sync conv=nocreat

\sphinxAtStartPar
dd if=/tmp/ramdisk/blok of=/dev/null bs=1048576 iflag=nocache,sync conv=nocreat
1024+0 records in
1024+0 records out
1073741824 bytes (1.1 GB, 1.0 GiB) copied, 0.240446 s, 4.5 GB/s


\bigskip\hrule\bigskip


\sphinxAtStartPar
modprobe zram
echo 80G | tee /sys/block/zram0/disksize    (80G ramdisk)
mkfs.ext4 /dev/zram0  (make a filesystem)
mkdir /RAM   (create a mountingpoint)
mount /dev/zram0 /RAM

\sphinxAtStartPar
now you can use the /RAM directory (which will be gone after poweroff)

\sphinxstepscope


\section{sharing a directory for LXC}
\label{\detokenize{proxmox/share:sharing-a-directory-for-lxc}}\label{\detokenize{proxmox/share::doc}}
\sphinxAtStartPar
on the host (pve node) create a shared directory (jan): (created on M2 storage)
chown \sphinxhyphen{}R nobody:nogroup jan
chmod 777 jan

\sphinxAtStartPar
\sphinxhref{mailto:root@pve}{root@pve}:/etc/pve/lxc\#


\subsection{modify the lxc}
\label{\detokenize{proxmox/share:modify-the-lxc}}
\sphinxAtStartPar
create a directory /mnt/shared (which will be the mounting point)


\subsection{on PVE modify the lxc config}
\label{\detokenize{proxmox/share:on-pve-modify-the-lxc-config}}
\sphinxAtStartPar
add this:
(mp0: /mnt/pve/M2P2/jan,mp=/mnt/shared)

\sphinxAtStartPar
arch: amd64
cores: 4
features: nesting=1
hostname: decode
memory: 5120
net0: name=eth0,bridge=vmbr0,firewall=1,hwaddr=BC:24:11:82:79:94,ip=dhcp,type=veth
ostype: ubuntu
rootfs: local\sphinxhyphen{}lvm:vm\sphinxhyphen{}101\sphinxhyphen{}disk\sphinxhyphen{}0,size=20G
swap: 512
unprivileged: 1
mp0: /mnt/pve/M2P2/jan,mp=/mnt/shared

\sphinxstepscope


\section{moving a LXC container}
\label{\detokenize{proxmox/moving:moving-a-lxc-container}}\label{\detokenize{proxmox/moving::doc}}
\sphinxAtStartPar
I had a container on a logical volume SCSI and I wanted to move it to logical volume M2

\sphinxAtStartPar
Cloning?

\sphinxAtStartPar
The container shared a directory, and cloning and displacing would mess this up.

\sphinxAtStartPar
Solution:
backup \& restore from backup on other volume

\sphinxstepscope


\section{nfs network between linux containers}
\label{\detokenize{proxmox/nfs_netwerk:nfs-network-between-linux-containers}}\label{\detokenize{proxmox/nfs_netwerk::doc}}

\subsection{set up a bridge}
\label{\detokenize{proxmox/nfs_netwerk:set-up-a-bridge}}
\sphinxAtStartPar
A Linux bridge interface (commonly called vmbrX) is needed to connect guests to the underlying physical network. It can be thought of as a virtual switch which the guests and physical interfaces are connected to.


\subsection{define an extra network interface in range 10.0.0.X}
\label{\detokenize{proxmox/nfs_netwerk:define-an-extra-network-interface-in-range-10-0-0-x}}
\sphinxstepscope


\section{NFS Host and Client Setup in Proxmox}
\label{\detokenize{proxmox/nfs_proxmox_setup:nfs-host-and-client-setup-in-proxmox}}\label{\detokenize{proxmox/nfs_proxmox_setup::doc}}
\sphinxAtStartPar
This guide will explain how to set up an NFS (Network File Sharing) server and add it as a remote storage in Proxmox.
\begin{enumerate}
\sphinxsetlistlabels{\arabic}{enumi}{enumii}{}{.}%
\item {} 
\sphinxAtStartPar
\sphinxstylestrong{Install NFS Server}

\sphinxAtStartPar
First, log in to the LXC container or the machine where the NFS server will be hosted. Then update the package list and install NFS:

\begin{sphinxVerbatim}[commandchars=\\\{\}]
apt\PYGZhy{}get\PYG{+w}{ }update
apt\PYGZhy{}get\PYG{+w}{ }install\PYG{+w}{ }sudo\PYG{+w}{ }\PYGZhy{}y
sudo\PYG{+w}{ }apt\PYG{+w}{ }install\PYG{+w}{ }nfs\PYGZhy{}kernel\PYGZhy{}server
\end{sphinxVerbatim}

\sphinxAtStartPar
Once installed, create a shared folder:

\begin{sphinxVerbatim}[commandchars=\\\{\}]
sudo\PYG{+w}{ }mkdir\PYG{+w}{ }/home/sharedfolder
sudo\PYG{+w}{ }chmod\PYG{+w}{ }\PYG{l+m}{777}\PYG{+w}{ }/home/sharedfolder
\end{sphinxVerbatim}

\sphinxAtStartPar
Next, edit the \sphinxcode{\sphinxupquote{/etc/exports}} file to configure the shared directory for export:

\begin{sphinxVerbatim}[commandchars=\\\{\}]
sudo\PYG{+w}{ }nano\PYG{+w}{ }/etc/exports
\end{sphinxVerbatim}

\sphinxAtStartPar
Add the following line (adjust the IP address and folder accordingly):

\begin{sphinxVerbatim}[commandchars=\\\{\}]
/home/sharedfolder\PYG{+w}{ }\PYG{l+m}{192}.168.1.0/24\PYG{o}{(}rw,sync,no\PYGZus{}subtree\PYGZus{}check\PYG{o}{)}
\end{sphinxVerbatim}

\sphinxAtStartPar
Save the file and restart the NFS server:

\begin{sphinxVerbatim}[commandchars=\\\{\}]
sudo\PYG{+w}{ }exportfs\PYG{+w}{ }\PYGZhy{}ra
sudo\PYG{+w}{ }systemctl\PYG{+w}{ }restart\PYG{+w}{ }nfs\PYGZhy{}kernel\PYGZhy{}server
\end{sphinxVerbatim}

\item {} 
\sphinxAtStartPar
\sphinxstylestrong{Configure NFS Storage in Proxmox}

\sphinxAtStartPar
Now log into your Proxmox host (the machine that will receive the NFS storage) and navigate to:

\sphinxAtStartPar
\sphinxstyleemphasis{Datacenter \textgreater{} Storage \textgreater{} Add \textgreater{} NFS}

\noindent\sphinxincludegraphics{{firefox_7FrqsKGmm4}.png}

\sphinxAtStartPar
Here, enter the NFS server’s IP address and select the shared directory.

\noindent\sphinxincludegraphics{{firefox_85gotgwjhw}.png}

\sphinxAtStartPar
Choose the desired contents for the storage (ISO images, containers, backups, etc.) and click \sphinxstyleemphasis{Add}.

\item {} 
\sphinxAtStartPar
\sphinxstylestrong{Set Permissions on LXC Container (If Applicable)}

\sphinxAtStartPar
If the NFS share will be used in an LXC container, ensure that permissions for NFS usage are set correctly:

\noindent\sphinxincludegraphics{{firefox_1mRMoEMAtl}.png}

\sphinxAtStartPar
Check the \sphinxstyleemphasis{NFS} box under \sphinxstyleemphasis{Options} for the LXC container.

\end{enumerate}

\sphinxAtStartPar
That’s it! You have successfully set up an NFS server and added it to Proxmox as remote storage.

\sphinxstepscope


\section{nfs client}
\label{\detokenize{proxmox/nfs_client:nfs-client}}\label{\detokenize{proxmox/nfs_client::doc}}
\sphinxAtStartPar
apt install nfs\sphinxhyphen{}common

\sphinxAtStartPar
mount \sphinxhyphen{}t nfs 10.0.0.104:/share /mnt/nfson104


\subsection{under node pve}
\label{\detokenize{proxmox/nfs_client:under-node-pve}}
\sphinxAtStartPar
watch out for features

\sphinxAtStartPar
rch: amd64
cores: 32
\sphinxstyleemphasis{features: mount=nfs,nesting=1}
hostname: llama
memory: 64000
net0: name=eth0,bridge=vmbr0,firewall=1,hwaddr=BC:24:11:58:0C:3A,ip=dhcp,type=veth
net1: name=eth5,bridge=vmbr1,firewall=1,hwaddr=BC:24:11:3A:F0:4A,ip=10.0.0.100/24,type=veth
ostype: ubuntu
rootfs: local\sphinxhyphen{}lvm:vm\sphinxhyphen{}100\sphinxhyphen{}disk\sphinxhyphen{}0,size=40G
swap: 512

\sphinxstepscope


\section{Portainer}
\label{\detokenize{proxmox/portainer:portainer}}\label{\detokenize{proxmox/portainer::doc}}

\subsection{using a script}
\label{\detokenize{proxmox/portainer:using-a-script}}
\sphinxAtStartPar
use a template for the linux container

\sphinxAtStartPar
a script from :
\sphinxurl{https://raw.githubusercontent.com/tteck/Proxmox/refs/heads/main/install/docker-install.sh}

\sphinxAtStartPar
systemctl start docker
systemctl status docker

\sphinxAtStartPar
\sphinxurl{https://192.168.0.182:9443} (your IP)


\subsection{by hand}
\label{\detokenize{proxmox/portainer:by-hand}}\begin{itemize}
\item {} 
\sphinxAtStartPar
create CT ubuntu22 with template

\item {} 
\sphinxAtStartPar
apt update

\item {} 
\sphinxAtStartPar
sudo apt install docker.io \sphinxhyphen{}y

\item {} 
\sphinxAtStartPar
sudo systemctl status docker

\item {} 
\sphinxAtStartPar
sudo usermod \sphinxhyphen{}aG docker \$USER (add current logged on user to docker group)

\item {} 
\sphinxAtStartPar
docker pull portainer/portainer\sphinxhyphen{}ce:latest

\item {} 
\sphinxAtStartPar
docker run \sphinxhyphen{}d \sphinxhyphen{}p 9000:9000 \textendash{}restart always \sphinxhyphen{}v /var/run/docker.sock:/var/run/docker.sock portainer/portainer\sphinxhyphen{}ce:latest

\end{itemize}


\subsection{exporting \& importing}
\label{\detokenize{proxmox/portainer:exporting-importing}}
\sphinxAtStartPar
this seems to work between systems:

\sphinxAtStartPar
(origin) sudo docker save ollama/ollama:latest  \textgreater{} my\sphinxhyphen{}ollama.tar
(target) sudo docker load \textless{} my\sphinxhyphen{}ollama.tar

\sphinxstepscope


\section{server for blender}
\label{\detokenize{proxmox/blender:server-for-blender}}\label{\detokenize{proxmox/blender::doc}}
\sphinxAtStartPar
Although the hp dl380 was 5 times faster than my i7 laptop, it is still slow, takes half an hour to render a picture.

\sphinxAtStartPar
One can adjust setting within blender to speed up things a bit, but still …


\subsection{howto render?}
\label{\detokenize{proxmox/blender:howto-render}}\begin{itemize}
\item {} 
\sphinxAtStartPar
download blender 4.2

\end{itemize}

\sphinxAtStartPar
cd blender\sphinxhyphen{}4.2.0\sphinxhyphen{}linux\sphinxhyphen{}x64/

\sphinxAtStartPar
./blender \sphinxhyphen{}b /home/naj/misvormde\sphinxhyphen{}donut1.blend \sphinxhyphen{}E CYCLES \sphinxhyphen{}f 1


\subsection{faster / less good}
\label{\detokenize{proxmox/blender:faster-less-good}}
\sphinxAtStartPar
./blender \sphinxhyphen{}b /home/naj/misvormde\sphinxhyphen{}donut1.blend \sphinxhyphen{}E BLENDER\_EEVEE\_NEXT \sphinxhyphen{}f 1


\subsection{at the movies}
\label{\detokenize{proxmox/blender:at-the-movies}}
\sphinxAtStartPar
./blender \sphinxhyphen{}b /home/naj/misvormde\sphinxhyphen{}donut1.blend \sphinxhyphen{}E BLENDER\_EEVEE\_NEXT \sphinxhyphen{}s 10 \sphinxhyphen{}e 500 \sphinxhyphen{}t 2 \sphinxhyphen{}a
./blender \sphinxhyphen{}b /home/naj/misvormde\sphinxhyphen{}donut1.blend \sphinxhyphen{}E BLENDER\_EEVEE\_NEXT \sphinxhyphen{}s 1 \sphinxhyphen{}e 100 \sphinxhyphen{}t 2 \sphinxhyphen{}a

\sphinxstepscope


\section{ollama install and use}
\label{\detokenize{proxmox/ollama:ollama-install-and-use}}\label{\detokenize{proxmox/ollama::doc}}
\sphinxAtStartPar
curl \sphinxhyphen{}fsSL \sphinxurl{https://ollama.com/install.sh} | sh


\subsection{using a container}
\label{\detokenize{proxmox/ollama:using-a-container}}
\sphinxAtStartPar
ollama\sphinxhyphen{}model\sphinxhyphen{}gemma2 was mounted using a volume and the image exported

\sphinxAtStartPar
sudo docker import ollama.tar ollama:latest

\sphinxstepscope


\section{configure elasticview}
\label{\detokenize{proxmox/elastic:configure-elasticview}}\label{\detokenize{proxmox/elastic::doc}}

\subsection{elasticsearch password}
\label{\detokenize{proxmox/elastic:elasticsearch-password}}
\sphinxAtStartPar
connect with term to container:
bin/elasticsearch\sphinxhyphen{}reset\sphinxhyphen{}password \sphinxhyphen{}u elastic
bin/elasticsearch\sphinxhyphen{}reset\sphinxhyphen{}password \sphinxhyphen{}u elastic \sphinxhyphen{}i (this allows for setting it yourself)


\subsection{testing elasticview connection}
\label{\detokenize{proxmox/elastic:testing-elasticview-connection}}
\sphinxAtStartPar
curl \sphinxhyphen{}X GET \sphinxhyphen{}k \sphinxhyphen{}u elastic:ebktuBhBHtE7N+JeBbIV “https://192.168.0.121:9200/\_cluster/health/?pretty”
\begin{description}
\sphinxlineitem{\{}
\sphinxAtStartPar
“cluster\_name” : “docker\sphinxhyphen{}cluster”,
“status” : “green”,
“timed\_out” : false,
“number\_of\_nodes” : 1,
“number\_of\_data\_nodes” : 1,
“active\_primary\_shards” : 1,
“active\_shards” : 1,
“relocating\_shards” : 0,
“initializing\_shards” : 0,
“unassigned\_shards” : 0,
“delayed\_unassigned\_shards” : 0,
“number\_of\_pending\_tasks” : 0,
“number\_of\_in\_flight\_fetch” : 0,
“task\_max\_waiting\_in\_queue\_millis” : 0,
“active\_shards\_percent\_as\_number” : 100.0

\end{description}

\sphinxAtStartPar
\}


\subsection{getting website certificate}
\label{\detokenize{proxmox/elastic:getting-website-certificate}}
\sphinxAtStartPar
connect to \sphinxurl{http://192.168.0.121}

\sphinxAtStartPar
firefox/settings/privicy\&security/certificates to add exception

\sphinxstepscope


\section{mounting}
\label{\detokenize{proxmox/mounting:mounting}}\label{\detokenize{proxmox/mounting::doc}}
\sphinxAtStartPar
mkdir /SCSIdata for the data on the scsidisk /dev/sdb
mkdir /M2data for the data on the M2 disk /dev/nvme0n1p2

\sphinxAtStartPar
\# mount /dev/sdb /SCSIdata

\sphinxAtStartPar
\# mount /dev/nvme0n1p2 /M2data


\section{edit  /etc/fstab}
\label{\detokenize{proxmox/mounting:edit-etc-fstab}}
\sphinxAtStartPar
/dev/nvme0n1p2  /M2data  ext4  defaults  0  2
/dev/sdb  /SCSIdata  ext4  defaults  0  2

\sphinxstepscope


\section{Backup pve}
\label{\detokenize{proxmox/backuppve:backup-pve}}\label{\detokenize{proxmox/backuppve::doc}}
\sphinxAtStartPar
Proxmox node now start from a sata m2 in the CDrom slot

\sphinxAtStartPar
A tar copy is made to M2data from /etc directory

\sphinxAtStartPar
In case of crash : reinstall proxmox on sata disk en copy backup.tar to /etc

\sphinxstepscope


\section{RAID}
\label{\detokenize{proxmox/raid:raid}}\label{\detokenize{proxmox/raid::doc}}
\sphinxAtStartPar
previously 2 600GB SAS disks were in RAID1 in the same volumegroup.

\sphinxAtStartPar
I have no spare disk, nor do I want to spend money on old tech.

\sphinxAtStartPar
Reconfigure : each disk is within own volume group, no more RAID, more (free) space.
Tool would not let me otherwise.


\subsection{No more raid :}
\label{\detokenize{proxmox/raid:no-more-raid}}
\sphinxAtStartPar
machine on one disk are backupped to other disk, and vice versa
one disk remains VG (volume group) within proxmox : used for LXC en VM images, backup
other disk contains ext4 filesystem and is a directory

\sphinxstepscope


\section{melborp server}
\label{\detokenize{proxmox/melborp:melborp-server}}\label{\detokenize{proxmox/melborp::doc}}

\subsection{solution for problem running pgadmin container and nginx}
\label{\detokenize{proxmox/melborp:solution-for-problem-running-pgadmin-container-and-nginx}}
\sphinxAtStartPar
Configure Nginx: Create a new configuration file for your domain in /etc/nginx/sites\sphinxhyphen{}available/melborp.solutions:

\sphinxAtStartPar
nginx
\begin{description}
\sphinxlineitem{server \{}
\sphinxAtStartPar
listen 80;
server\_name www.melborp.solutions;
\begin{description}
\sphinxlineitem{location / \{}
\sphinxAtStartPar
proxy\_pass \sphinxurl{http://localhost:8888};  \# Forward traffic to the pgAdmin container
proxy\_set\_header Host \$host;
proxy\_set\_header X\sphinxhyphen{}Real\sphinxhyphen{}IP \$remote\_addr;
proxy\_set\_header X\sphinxhyphen{}Forwarded\sphinxhyphen{}For \$proxy\_add\_x\_forwarded\_for;
proxy\_set\_header X\sphinxhyphen{}Forwarded\sphinxhyphen{}Proto \$scheme;

\end{description}

\sphinxAtStartPar
\}

\end{description}

\sphinxAtStartPar
\}


\chapter{Indices and tables}
\label{\detokenize{proxmox/index:indices-and-tables}}\begin{itemize}
\item {} 
\sphinxAtStartPar
\DUrole{xref}{\DUrole{std}{\DUrole{std-ref}{genindex}}}

\item {} 
\sphinxAtStartPar
\DUrole{xref}{\DUrole{std}{\DUrole{std-ref}{modindex}}}

\item {} 
\sphinxAtStartPar
\DUrole{xref}{\DUrole{std}{\DUrole{std-ref}{search}}}

\end{itemize}



\renewcommand{\indexname}{Index}
\printindex
\end{document}