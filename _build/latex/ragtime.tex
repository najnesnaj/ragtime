%% Generated by Sphinx.
\def\sphinxdocclass{report}
\documentclass[letterpaper,10pt,english]{sphinxmanual}
\ifdefined\pdfpxdimen
   \let\sphinxpxdimen\pdfpxdimen\else\newdimen\sphinxpxdimen
\fi \sphinxpxdimen=.75bp\relax
\ifdefined\pdfimageresolution
    \pdfimageresolution= \numexpr \dimexpr1in\relax/\sphinxpxdimen\relax
\fi
%% let collapsible pdf bookmarks panel have high depth per default
\PassOptionsToPackage{bookmarksdepth=5}{hyperref}

\PassOptionsToPackage{warn}{textcomp}
\usepackage[utf8]{inputenc}
\ifdefined\DeclareUnicodeCharacter
% support both utf8 and utf8x syntaxes
  \ifdefined\DeclareUnicodeCharacterAsOptional
    \def\sphinxDUC#1{\DeclareUnicodeCharacter{"#1}}
  \else
    \let\sphinxDUC\DeclareUnicodeCharacter
  \fi
  \sphinxDUC{00A0}{\nobreakspace}
  \sphinxDUC{2500}{\sphinxunichar{2500}}
  \sphinxDUC{2502}{\sphinxunichar{2502}}
  \sphinxDUC{2514}{\sphinxunichar{2514}}
  \sphinxDUC{251C}{\sphinxunichar{251C}}
  \sphinxDUC{2572}{\textbackslash}
\fi
\usepackage{cmap}
\usepackage[T1]{fontenc}
\usepackage{amsmath,amssymb,amstext}
\usepackage{babel}



\usepackage{tgtermes}
\usepackage{tgheros}
\renewcommand{\ttdefault}{txtt}



\usepackage[Bjarne]{fncychap}
\usepackage{sphinx}

\fvset{fontsize=auto}
\usepackage{geometry}


% Include hyperref last.
\usepackage{hyperref}
% Fix anchor placement for figures with captions.
\usepackage{hypcap}% it must be loaded after hyperref.
% Set up styles of URL: it should be placed after hyperref.
\urlstyle{same}

\addto\captionsenglish{\renewcommand{\contentsname}{Contents:}}

\usepackage{sphinxmessages}
\setcounter{tocdepth}{1}



\title{ragtime}
\date{Oct 18, 2024}
\release{1}
\author{jan jansen}
\newcommand{\sphinxlogo}{\vbox{}}
\renewcommand{\releasename}{Release}
\makeindex
\begin{document}

\pagestyle{empty}
\sphinxmaketitle
\pagestyle{plain}
\sphinxtableofcontents
\pagestyle{normal}
\phantomsection\label{\detokenize{index::doc}}



\chapter{Intro}
\label{\detokenize{intro:intro}}\label{\detokenize{intro::doc}}
\sphinxAtStartPar
These are some building blocks in what should become a local LLM for financial info.
I do like investing, but I rather not read all the publications.

\sphinxAtStartPar
This software should be able to  :
\begin{itemize}
\item {} 
\sphinxAtStartPar
leech reports

\item {} 
\sphinxAtStartPar
embed them into a vectordb

\item {} 
\sphinxAtStartPar
use a local LLM to retrieve and bundle information

\item {} 
\sphinxAtStartPar
generate a report

\end{itemize}


\section{Homelab: (did not want to create separate repo)}
\label{\detokenize{intro:homelab-did-not-want-to-create-separate-repo}}
\sphinxAtStartPar
I have an old HP dl380p gen8, which I modified:
\sphinxhyphen{} removed a SAS controller card, and have disks on internal controller
\sphinxhyphen{} add a 1TB NVME M2 disk, on PCI\sphinxhyphen{}e 3 adapter
\sphinxhyphen{} installed Proxmox

\sphinxAtStartPar
todo:
\sphinxhyphen{} boot on sata disk on cdrom port
\sphinxhyphen{} remove RAID1 and use disks as such: space over security
\sphinxhyphen{} insert NVIDIA P4 single slot


\chapter{install something from docker}
\label{\detokenize{docker:install-something-from-docker}}\label{\detokenize{docker::doc}}\begin{quote}

\sphinxAtStartPar
\sphinxurl{https://docs.docker.com/engine/install/ubuntu/}
\sphinxurl{https://download.docker.com/linux/ubuntu/dists/jammy/pool/stable/amd64/}
\end{quote}


\section{using portainer:}
\label{\detokenize{docker:using-portainer}}\begin{quote}

\sphinxAtStartPar
sudo docker volume create portainer\_data
sudo docker run \sphinxhyphen{}d \sphinxhyphen{}p 8000:8000 \sphinxhyphen{}p 9443:9443 \textendash{}name portainer \textendash{}restart=always \sphinxhyphen{}v /var/run/docker.sock:/var/run/docker.sock \sphinxhyphen{}v portainer\_data:/data portainer/portainer\sphinxhyphen{}ce:2.21.3
\end{quote}


\section{leecher:}
\label{\detokenize{docker:leecher}}
\sphinxAtStartPar
custommade : pdf copy to directory leech


\section{building pdf to text:}
\label{\detokenize{docker:building-pdf-to-text}}
\sphinxAtStartPar
docker build \sphinxhyphen{}t pdf\sphinxhyphen{}text\sphinxhyphen{}converter .
docker run \sphinxhyphen{}v \$(pwd)/leech:/app/leech \sphinxhyphen{}v \$(pwd)/pages:/app/pages pdf\sphinxhyphen{}text\sphinxhyphen{}converter


\section{pdfconverter2}
\label{\detokenize{docker:pdfconverter2}}
\sphinxAtStartPar
docker build \sphinxhyphen{}t pdf\sphinxhyphen{}watcher .
docker run \sphinxhyphen{}v (\$pwd)/leech:/leech \sphinxhyphen{}v (\$pwd)/pages:/pages pdf\sphinxhyphen{}watcher


\chapter{storing vector + metadata}
\label{\detokenize{pgvector:storing-vector-metadata}}\label{\detokenize{pgvector::doc}}
\sphinxAtStartPar
PostgreSQL + pgvector Example
If you prefer using PostgreSQL, you can use the pgvector extension, which allows you to store vector embeddings in a PostgreSQL table alongside metadata.

\sphinxAtStartPar
Steps:
\begin{itemize}
\item {} 
\sphinxAtStartPar
Install pgvector: First, install the pgvector extension.

\item {} 
\sphinxAtStartPar
Create a table: Create a table that stores both vectors and metadata.

\item {} 
\sphinxAtStartPar
Insert vectors and metadata: Insert each chunk’s vector along with its metadata.

\end{itemize}

\sphinxAtStartPar
sql


\chapter{A poor man’s rig}
\label{\detokenize{poorman:a-poor-man-s-rig}}\label{\detokenize{poorman::doc}}
\sphinxAtStartPar
I theory it should be possible to use cheap hardware.

\sphinxAtStartPar
I have an old server (proliant gen8), but it has 256GB of RAM.
I theory I can store a big LLM on RAMDISK and use this trick:

\sphinxAtStartPar
\sphinxurl{https://huggingface.co/blog/lyogavin/run-llama-405b-on-4gb-vram}



\renewcommand{\indexname}{Index}
\printindex
\end{document}